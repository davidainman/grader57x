\documentclass[12pt]{article}
\renewcommand{\labelenumi}{\arabic{enumi}.} 
\renewcommand{\labelenumii}{\arabic{enumi}.\arabic{enumii}}
\renewcommand{\labelenumiii}{\arabic{enumi}.\arabic{enumii}.\arabic{enumiii}}
\usepackage{enumitem}
\setlist{  
  listparindent=\parindent,
  parsep=0pt,
}
\begin{document}

\title{Homework Guidelines for 57x}
\author{David Inman}
\date{\today}
\maketitle

\tableofcontents{}

\section{Introduction}
This document is meant to guide professors and TAs when creating and grading homework assignments for the 57x series. It is roughly structured according to the homework workflow: creating the homework description (\S\ref{sec:homework}), creating the rubric (\S\ref{sec:rubric}), creating the necessary files for grading (\S\ref{sec:files}), setting up the homework checker for student use (\S\ref{sec:checker}), and the use of the automated grader (\S\ref{sec:autograder}). Finally, I have included some documentation for the structure of the grader and checker (\S\ref{sec:documentation}) and a list of items for future development (\S\ref{sec:future}).


\section{Homework Descriptions} \label{sec:homework}
There are a series of common mistakes students make when doing the homework for the 57X series. This is a set of best practices when writing the homework descriptions to minimize student confusion and errors, and to make the homework easier to grade. The development of homework descriptions is presumed to be done by the professor.

\begin{enumerate}
\item {Descriptions of homework should be made as clear as possible. 

It is sometimes difficult to fully clarify an assignment. When writing an assignment description, the best practice for clarity is to separate different parts of the assignment by section. If possible, describe the expected function of the program (Viterbi, classification, etc) in one section, and then describe the format of the output in the following section or a separate sub-section. Try to minimize component interleaving in the description, although full separation may be impossible. }

\item {Homeworks should include an at-a-glance summary of turned-in files. }

\item {Points should be given at section headers in the homework description. These should add up to 75, the portion of the rubric assigned for homework-specific grading (\S\ref{sec:specificrubric}). If these points are sub-divided within each section (e.g., you want to specify the number of points given for the code finishing in a period of time, or falling within an acceptable parameter), those points should be given at the top of a sub-section. }

\item {Points should be awarded for program output, not for code structure.
  
  This prevents the TA from needing to review the code for each student. For instance, ``Program outputs a valid path (10 points)" is preferable to ``Initializes lattice (10 points)."}
 
\item {Graded outputs should have a strict format.
  
  If a portion of a student's output is to be graded, that output must be uniform across all students. This means that the output must be very precisely formulated. For instance, ``Epsilon arcs are followed without printing an output value (10 points)" is preferable to ``Handles epsilon transitions (10 points)." Likewise, ``Re-print the input string with the labeled parts of speech after each word, followed by a tab character: e.g., word1 {\textbackslash}t POS1 {\textbackslash}t word2 {\textbackslash}t POS2 (...)" is better than ``Print the input string with the parts of speech after each word." }
 
 \item {Graded numeric values should be written to a separate output file, when possible.

 If students need to generate a table of values based on running their code, the TA has to verify the validity of it twice: In the table in the readme and in the output code. While it is worthwhile for students to collate data into a presentable format, it will make things easier to grade if the value present in the table is easily extracted from one of the output files when running the program. That is, ``Table and {\textless}output file{\textgreater} show correct accuracy and precision trends" is preferable to ``Table shows correct trends." }
\end{enumerate}

\section{Rubric} \label{sec:rubric}

The rubric across 570, 571, and 572 has been standardized. There are two parts: the standard component, which is graded identically across all assignments, and then the homework-specific portion, which is defined for each assignment. This section is presumed to be done by the TAs.

\subsection{Standard Portion (25 points)} \label{sec:standardrubric}

There is no deviation for this component, and it is graded automatically by the grader (\S\ref{sec:autograder}). It is graded as follows:

\begin{enumerate}
\item Submitted Files (10 points)
  \begin{enumerate}
    \item Tarfile submitted (2 points)
    \begin{enumerate}
      \item Exists (1 point)
      \item Correctly named (1 point)
    \end{enumerate}
    \item Readme file submitted (2 points)
    \begin{enumerate}
      \item Exists (1 point)
      \item Correctly named (1 point)
    \end{enumerate}
    \item Requested files exist in tarfile (6 points)
  \end{enumerate}
  \item Program runs as expected (15 points)
  \begin{enumerate}
    \item Code runs to completion on input (10 points)
    \item Output of running code matches turned-in files (5 points)
  \end{enumerate}
\end{enumerate}

\subsection{Homework-specific Portion (75 points)} \label{sec:specificrubric}

This section cannot be schematized, but is based on the needs of each assignment. However, there are some best practices. Although the gross distribution of points is defined by the professor in the homework description, the TA still has to make decisions about the finer distribution.

\begin{enumerate}
\item {Some points should be assigned for correct formatting for each output file.}
 
\item {Points distributed for the program being implemented correctly should depend on evaluating the program output on test files that are as modular as possible.

For instance, the grader should create an input that will test for the code's ability to handle multiple labeling paths, and assign points for ``Chooses the correct labeling path" based on that test. Then a separate test may look for the code handling unknown words. These tests should not be intermixed (or there should be a separate test input that mixes them). }
 
\end{enumerate}


\section{Creating Files for Grading} \label{sec:files}

\subsection{List of files needed} \label{sec:fileslist}

For each assignment, a series of files needs to be generated for the purposes of grading. This is an attempt to list all the necessary files. The first file (submit-file-list) is suggested to be the professor's responsibility, and the rest are suggested to be the TA's.

\begin{enumerate}
\item {submit-file-list

This is a file that lists all the expected files in the student submission. For details, see \S\ref{sec:submit-list}.
}
\item {config\_hwX

This is the config file read by the automated grader. It is explained in more detail in \S\ref{sec:config}.
}
\item {run\_hwX.sh

This is a file that will run the student's code. When the grader runs student files, it calls this shell script. It is possible to have multiple run files in an assignment: for example, if one needs to run within a certain time limits and others do not.

This script will be called from the top directory of the student's untarred submit file, so all paths should be relative to that location.

The run file(s) needs to serve two functions:

1. It should rerun the commands that generate output the student turned in. These new output files should be put in the same location as the turned-in files with the same name, with a \texttt{.out} appended. For example, if the student turns in a file at \texttt{Q2/lambda1.lm}, the run file should create a file \texttt{Q2/lambda1.lm.out}.

2. It should run any commands that run student code on hidden test input.
}
\item {Gold files

For every student output, there must be a gold output file. For clarity's sake, I suggest giving them the extension \texttt{.gold}. If you have withheld input or test data, there also must be gold output for those cases.
}
\item {Withheld test input (optional)

If you are running student code on withheld test input (e.g., modular or targeted tests), these files need to be written.

}
\end{enumerate}

Once all these files have been generated and tested on a homework, this set of files should be saved and passed on for future courses. The submit-file-list (1) needs to be made available to students for homework checker, so should be put on \texttt{/dropbox} with the assignment. The suggestion for the rest is to put the rest into a TA/instructor-only grading folder for the assignment, with the config and run files (2, 3) in the top directory, gold files (4) into a folder called \texttt{gold}, any test files (5) into a folder called \texttt{test}.

\subsection{Submit List} \label{sec:submit-list}

The submit-file-list is a file that lists all the files expected in the student's submitted tar file. The format is a set of newline-separated list of expected files, in relative path format. For instance, the file might look like the following:

\vspace{5pt}

\texttt{run.sh}

\texttt{table}

\texttt{Q2/q2.sh}

\texttt{Q2/decoder.sh}

\texttt{Q3/q3.sh}

\texttt{Q3/results/data.txt}

\subsection{Config File} \label{sec:config}

The config file is a plaintext file with configuration inputs separated by line. The format is similar to condor .cmd files, and there is no required order to the config file: each line stands on its own. The format and configuration lines are:

\vspace{5pt}

1. \texttt{zipfile = <zipfile name>}

\texttt{<zipfile name>} is the full path of the zip file that was generated by Canvas when you downloaded the assignments.

\vspace{5pt}

2. \texttt{destination = <destination folder name>}

\texttt{<destination folder name>} is the local path to the folder where student submissions will be unzipped into (for unzipping), or where the student submissions already exist (for evaluating).

\vspace{5pt}

3. \texttt{report = <name of report file>}

\texttt{<name of report file>} is the local path to the report file. This line is optional: If it is not present, the report file will be given the name \texttt{<destination folder>\_report.txt}.

\vspace{5pt}

4. \texttt{file\_structure = <submit-file-list>}

\texttt{<submit-file-list>} is the same file that the automated checker uses and that the professor creates for the assignment (see \ref{sec:submit-list}).

\vspace{5pt}

\noindent The following config lines can be repeated multiple times:

5. \texttt{repro\_comparison = <student\_file> <sorted|unsorted>}

This line should be repeated for as many student output files as the assignment requires. The grader will compare the file located at the \textit{relative} path \texttt{<student\_file>} to the expected file at \texttt{<student\_file>.out} (which should be generated by \texttt{run\_hwX.sh}).

You must specify whether the file should be \texttt{sorted} or \texttt{unsorted} before comparison occurs. If the order of the output file is unimportant (for instance, it is the unordered output from a hash), then it is best to select \texttt{sorted}.

\vspace{5pt}

6. \texttt{gold\_comparison = <gold\_file> <student\_file> <sorted|unsorted>}

This line should be repeated for as many student output files as are being graded, including output that was generated by \texttt{run\_hwX.sh}. \texttt{<gold\_file>} is the \textit{absolute} path to the gold file, and \texttt{<student\_file>} is the \textit{relative} path to the student file.

As with \texttt{repro\_comparison}, you must specify whether the files should first be \texttt{sorted} before comparing, or left \texttt{unsorted}.

\vspace{5pt}

\noindent The following config lines are principally for future use (see \S\ref{sec:future}), and are not necessary for the functioning of the current program.

\vspace{5pt}

7. \texttt{total\_grade = <total points available for automated section>}

  This is an integer value of how many points are reported in the automatic portion of the grading. The number should always be at least 25 (which is the standard portion). If part of the grading has to be done manually, this value will be less than 100.
  
  This line is optional. If it is not in the config file, the default value is 25 (i.e., only the standard grade will be assigned). Currently, the grader will not assign point values to items defined in the gold\_comparison section (\#6 above).

\vspace{5pt}

8. \texttt{run\_script = <run\_hwX.sh>}

This is the place to point the config file to the \texttt{run\_hwX.sh} file described in \S\ref{sec:fileslist}. Currently this is executed by a separate wrapper.

\vspace{5pt}

9. \texttt{cmd = <condor cmd file>}

\texttt{<condor cmd file>} is a generic condor file that is used to run \texttt{run\_script} for all students. It should probably never be changed from the \texttt{hw.cmd} file present in the grader directory. As with \#8, this is currently executed by a separate wrapper.

\vspace{5pt}

\section{Automated Checker (check\_hw.py)} \label{sec:checker}

The automated checker is for use by students. It confirms that their hw.tar.gz file has all expected files. It runs the same file-checking that the full grader does, but also checks for expected source code. The method for calling the code is:

\texttt{python check\_hw.py <languages> <submit-file-list> <tar\_name>}

\subsection{Checker dependencies}

\begin{enumerate}
\item {\texttt{<languages>}

This is a file that contains a list of recommended coding languages and their executable formats. The format is line-separated, and each line lists an executable file format, followed by the source code formats. That is, each line looks like:

\texttt{executable\_extension code\_extension1 code\_extension2 ...}

If a language is scripted, only the extension of the script appears, and there may be multiple lines with different executables. An example file could look like:

\vspace{5pt}

\texttt{exe cs}

\texttt{exe c h}

\texttt{class java}

\texttt{jar java}

\texttt{py}

\texttt{pl}

\vspace{5pt}

The master version of this file is called ``languages" and lives in the same directory as the check\_hw.py script.

}
\item {\texttt{<submit-file-list>}

This is the same file described in \ref{sec:submit-list}.

}
\item {\texttt{<tar\_name>}

This is the name of the tar file the homework checker looks for. Following current 57x policy, it should be hw.tar.gz.
}
\end{enumerate}

\subsection{Student-facing check\_hw.sh}

The check\_hw.py script is extensible and configurable. This is for flexibility for us, but not ideal for student use. The current policy is to create a wrapper, called check\_hw.sh, inside the directory of each homework assignment. This shell script calls check\_hw.py with the correct parameters for the given assignment.

\subsection{Behavior}

The check\_hw script performs the following checks:

\begin{itemize}
\item Checks to see if the tar extracts. If it doesn't, prints an error.
\item Checks for all files and folder structures listed in \texttt{submit-file-list}. Prints an error for each file and folder missing.
\item Checks that all *.sh files begin with \#!/bin/sh. Prints an error for each *.sh file that doesn't.
\item Checks to see if there is some code from a language listed in \texttt{languages}. If not, prints a warning.
\item If the tar includes binaries, checks for (expected) corresponding source code. Prints a warning if not found. If found, it prints the binary file and its (assumed) source code.
\item If the script runs into an exception, it prints out some information and asks the user to email davinman@uw.edu with the tar file that triggered the exception.
\end{itemize}

\section{Automated Grader (grader.sh)} \label{sec:autograder}

There are three steps to using the automatic grader: unpacking the zip file from Canvas and grading the presence of requested files in the submission (\ref{sec:unzipping}), running student code (\ref{sec:runningcode}), and grading the run code (\ref{sec:secondgrade}). Most of these are run through \texttt{grader.sh}. \texttt{grader.sh} is always run with the following command:

\vspace{5pt}

\texttt{grader.sh <config\_file> <parameters>}

\vspace{5pt}

\noindent Each stage of running the grader requires different parameters. For more on the config file, see \S\ref{sec:config}.

\subsection{Unzipping the zip file and file checking} \label{sec:unzipping}

To open the zip file, use the parameter \texttt{open} when calling \texttt{grader.sh}. This will expand the zip file and all students' tar files into the destination folder defined in \texttt{config\_file}.

The destination folder will have student submissions sorted into folders based on the student's name as given by Canvas. Each folder will be a concatenated string of the student's last, first, and middle names (in that order), and within each folder, the submitted tar file will be exploded.

Upon opening student folders, the grader will grade the ``Submitted Files" section of the standard portion of the rubric (\S\ref{sec:standardrubric}). This information will be written out to the report file defined in \texttt{config\_file}, which is sorted alphabetically by student name.

TODO: If a student submits their code late or separately, use the parameter \texttt{open\_student} followed by the student's name and their tar file. An example of calling this parameter is:

\vspace{5pt}

\texttt{grader.sh hw1\_config open\_student doejohn john\_hw.tar.gz}

\vspace{5pt}

This will add the contents of \texttt{john\_hw.tar.gz} to the folder containing other student submissions, under the subdirectory \texttt{doejohn}, and add their grade information into the report. This command will integrate John Doe's submission with other student submissions, as if it had existed in the original zip file.

One warning about this method: It cannot see if the student turned in the readme file, since readme files are turned in outside of the tar file. This script assumes the readme file is present and does not take off points.

\vspace{5pt}

\subsection{Running student code} \label{sec:runningcode}

Student code is run by calling:

\vspace{5pt}

\noindent \texttt{run\_all.sh <hw\_directory> <run\_script.sh>}

\vspace{5pt}

\noindent The \texttt{hw\_directory}  is the same as the \texttt{destination} directory in the config file (\S\ref{sec:config}), and \texttt{run\_script} is also the same as from the config file.

The code loops through every student directory in \texttt{hw\_directory} and calls the \texttt{run\_script} through condor within the directory. It will launch X condor commands where X is the number of students.

To run a single student's code, the command is:

\vspace{5pt}

\noindent \texttt{run\_student.sh <student\_directory\_path> <run\_script.sh>}

\vspace{5pt}

Note that \texttt{student\_directory\_path} must be the \textit{full} path to the student directory, e.g. \texttt{hw4/doejohn/}.

\subsection{Evaluation} \label{sec:secondgrade}

After the condor scripts have all finished (you must manually check with \texttt{condor\_q <username>}), you can run the evaluation component of the automatic grader. This is done by calling:

\vspace{5pt}

\noindent \texttt{grader.sh hw\_config eval\_all}

\vspace{5pt}

This will launch all the comparisons between run output and turned-in output, and between output and gold standard, that has been defined in the config file (\S\ref{sec:config}). If any differences are found in the comparisons, a summary of the number of differing lines will be written to the report file, and the diff itself will be written to an output file in the student directory.

To run the evaluation on a single student, the command is:

\vspace{5pt}

\noindent \texttt{grader.sh hw\_config eval\_student <student\_directory\_name>}

\vspace{5pt}

Note that \texttt{student\_directory\_name} is the name of the student directory itself (without any containing folder). That is, the parameter should be \texttt{doejohn}, not \texttt{hw4/doejohn}.

\section{Code Documentation} \label{sec:documentation}

The code for the automated grader is written in Python and located inside the grader57x folder.

TODO: This section will be filled out at a later date.

\section{Future Grader Development} \label{sec:future}

This section lists the most important needs (as I perceive them) for future development.

\subsection{Internalizing the run script} \label{sec:internalrun}

Currently, the run script is separate from the \texttt{grader.sh} script. This is due to some difficulties Python has calling into condor. It would be more parsimonious if these difficulties could be overcome, and running the assignment could be called from \texttt{grader.sh} with a \texttt{run\_all} and \texttt{run\_student} parameter, just like the other two grading steps.

\subsection{Assigning numerical values to diffing gold files} \label{sec:assignnumbers}

Currently the grader does not assign numeric values according to whether or how different a student's output file is from the gold standard. This will need to be implemented in order to implement (\ref{sec:speciallines}) and (\ref{sec:numericvalues}) below.

\subsection{Grading special lines in the output} \label{sec:speciallines}

For some assignments, only particular lines of the output need to be graded. It would be good to add a parameter that, for certain comparison operations, would only grade lines that had a leading string in them.

\subsection{Grading numerical values in the output} \label{sec:numericvalues}

For some assignments, a numerical value must fall within a particular range. It would be good to add a parameter that looked for a number (perhaps with leading or trailing strings). The format for this should be: number, difference\_percent (where the two parameters are interpreted to mean a value must fall within the range \textit{number} $\pm$ \textit{difference\_percent}).

\end{document}
